\documentclass[11pt,a4paper]{article}
\usepackage[utf8]{inputenc}
\usepackage[margin=0.75in]{geometry}
\usepackage{multicol}
\usepackage{enumitem}
\usepackage{titlesec}
\usepackage{xcolor}
\usepackage{hyperref}

% Define colors
\definecolor{darkgray}{RGB}{80,80,80}
\definecolor{lightgray}{RGB}{120,120,120}

% Remove page numbering
\pagestyle{empty}

% Custom section formatting
\titleformat{\section}{\large\bfseries\color{darkgray}}{}{0em}{}[\titlerule]
\titleformat{\subsection}{\normalsize\bfseries\color{darkgray}}{}{0em}{}

% Adjust spacing
\titlespacing*{\section}{0pt}{12pt}{8pt}
\titlespacing*{\subsection}{0pt}{8pt}{4pt}

% Custom list formatting
\setlist[itemize]{leftmargin=15pt, itemsep=2pt, parsep=0pt, topsep=4pt}
\setlist[itemize,2]{leftmargin=20pt, itemsep=1pt, parsep=0pt, topsep=2pt}

% Hyperlink setup
\hypersetup{
	colorlinks=true,
	linkcolor=black,
	urlcolor=black,
	citecolor=black
}

\begin{document}
	
	% Header
	\begin{center}
		{\Huge\bfseries Dr. Argyrios Koumtzis}\\[8pt]
		Göttingen, Germany | +49 162 9583532 | \href{mailto:argiris.koumtzis@gmail.com}{argiris.koumtzis@gmail.com}
	\end{center}
	
	\vspace{10pt}
	
	% Two-column layout
	\begin{multicols}{2}
		
		% LEFT COLUMN
		\section{Skills}
		
		\subsection{Programming:}
		\begin{itemize}
			\item \textbf{Advanced:} C, Fortran, \textbf{Python} (NumPy, SciPy, Matplotlib, MayaVi, PyTorch, Scikit-Learn)
			\item \textbf{Intermediate:} Julia, Version Control (Git), Emacs Lisp, bash, high performance computing clusters
			\item \textbf{Beginner:} Parallel computing: OpenMP (used with Fortran and C), GPU computing using CUDA C, SQL
			\item \textbf{Computational modeling:} PDEs, finite differences, spectral methods
			\item \textbf{Machine learning:} Physics-informed neural networks optimization algorithms
			\item Data analysis, data visualization
			\item \textbf{Plasma physics:} Magnetohydrodynamics, kinetic simulations, anomalous transport phenomena
			\item \textbf{Office applications:} Microsoft Office, Libre Office, LaTeX, Emacs
		\end{itemize}
		
		\subsection{Languages}
		\begin{itemize}
			\item \textbf{Greek:} native
			\item \textbf{English:} professional
			\begin{itemize}
				\item TOEFL: 115/120
			\end{itemize}
		\end{itemize}
		
		\section{Finance \& Entrepreneurship}
		\begin{itemize}
			\item Developed Python code for stock screening
			\item Completed courses in Introduction to Corporate Finance at Columbia University via EDX and reading relevant books
			\item Managing a personal stock portfolio
			\item Financial modeling, e.g. DCF
			\item Advisor for a robotics team that won \textbf{first place} in the \textbf{International Robot Olympiad 2020}
			\item Proactively researching the real estate market to identify attractive investment opportunities
			\item Managing two Airbnb units
		\end{itemize}
		
		\columnbreak
		
		% RIGHT COLUMN
		\section{Summary}
		Computational researcher and quantitative developer with 8 years experience in tools like partial differential equations, \textbf{optimization algorithms}, Monte Carlo simulations, \textbf{machine learning} and physics informed neural networks. A team player who is also ready to take leadership roles and enjoys mentoring junior team members. Being a self-starter in finance, I am excited to apply my quantitative skills to explore the field of quantitative finance within dynamic fast-paced environments.
		
		\section{Research Experience \& Education}
		
		\subsection{July 2023 - present}
		\textbf{Postdoctoral Researcher | Max Planck Institute for Solar System Research}
		\begin{itemize}
			\item Implementing \textbf{physics-informed neural networks} for solving partial differential equations that characterize the magnetic field structure within the solar corona
			\item Validating computational models of the coronal magnetic field through comparative analysis with spacecraft observations
		\end{itemize}
		
		\subsection{October 2019 – June 2023}
		\textbf{PhD Candidate | Max Planck Institute for Solar System Research}
		\begin{itemize}
			\item Thesis topic: Computational modeling of the solar coronal magnetic field
			\begin{itemize}
				\item Grade: \textbf{magna cum laude}
			\end{itemize}
			\item \textbf{Key result:} Developed a \textbf{state-of-the-art numerical code} that reconstructs the 3D structure of the solar coronal magnetic field using observations from space telescopes
			\item The code is written in C language, parallelized with open MP, and implemented on a Yin-Yang computational grid
			\item Participating in a multi-institution international team formed to compare coronal magnetic field models
			\item \textbf{Teaching experience:} Supervision of junior team members
		\end{itemize}
		
		\subsection{2014 – 2019}
		\textbf{Bachelor's Degree in Physics | Aristotle University of Thessaloniki, Greece}
		\begin{itemize}
			\item \textbf{GPA: 9.39/10} (ranking 1st among 280 students)
			\item \textbf{Undergraduate Research Assistant} in group Plasma Physics \& High Energy Astrophysics | 01/2016 – 10/2019
			\item Thesis: Power law flare statistics driven by photospheric turbulence
			\item Developed a \textbf{spectral code} for linear force-free magnetic field extrapolation in the solar corona in Fortran and Julia
			\item Developed a Fortran \textbf{Monte Carlo code} driven by observed probability distributions to study solar flare statistics
			\item Teaching assistant in Astronomy course
		\end{itemize}
		
		\subsection{Summer research intern | Oxford University, UK | Summer 2017}
		\begin{itemize}
			\item Estimating intergalactic magnetic fields
			\item Conducted 4D high performance computing Monte Carlo simulations
			\item Used \textbf{machine learning} to analyze multidimensional simulation data output
		\end{itemize}
		
	\end{multicols}
	
	\section{Conferences}
	\begin{itemize}
		\item \textbf{Speaker in}
		\begin{itemize}
			\item EGU (European Geological Union) | 2022, 2023
			\item DPG (Deutsche Physikalische Gesellschaft) Meeting | 2022
			\item Workshop "Nonlinear Dynamical Systems" | 2019
		\end{itemize}
		\item \textbf{Invited participant:} Interdisciplinary international workshop "The audible universe" 1 and 2 | 2021, 2022
	\end{itemize}
	
	\section{Scholarships \& Awards}
	\begin{itemize}
		\item Scholarship for academic excellence from National Bank of Greece | 2018
		\item Scholarship for an Educational Trip to \textbf{Massachusetts Institute of Technology} | 2016
		\item Award by the Greek Rotary Club for achieving a societal change: contribution to free access for the disabled to higher education | 2015
		\item 5th Position in International Youth Chess Championship for the blind | 2015
	\end{itemize}
	
	\section{Publications}
	\begin{itemize}
		\item Superdiffusive Stochastic Fermi Acceleration in Space and Energy, 2019, N. Siulas, H. Isliker, L. Vlahos, A. Koumtzis.
		\item A New Global Nonlinear Force-Free Coronal Magnetic-Field Extrapolation Code Implemented on a Yin-Yang Grid, 2023, A. Koumtzis, T. Wiegelmann.
		\item Global full sphere coronal field extrapolation during solar minimum and maximum, A. Koumtzis, T. Wiegelmann, M. Madjarska, to be submitted in November 2023.
	\end{itemize}
	
\end{document}