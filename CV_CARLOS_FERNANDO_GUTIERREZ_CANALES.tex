\documentclass[11pt,a4paper]{article}
\usepackage[utf8]{inputenc}
\usepackage[margin=0.75in]{geometry}
\usepackage{multicol}
\usepackage{enumitem}
\usepackage{titlesec}
\usepackage{xcolor}
\usepackage{fontawesome5}  % Add this in your preamble
\usepackage{hyperref}

% Define colors
\definecolor{darkpurple}{RGB}{188,0,128}
\definecolor{lightpurple}{RGB}{177,156,217}

% Remove page numbering
\pagestyle{empty}

% Custom section formatting
\titleformat{\section}{\large\bfseries\color{darkpurple}}{}{0em}{}[\titlerule]
\titleformat{\subsection}{\normalsize\bfseries\color{darkpurple}}{}{0em}{}

% Adjust spacing
\titlespacing*{\section}{0pt}{12pt}{8pt}
\titlespacing*{\subsection}{0pt}{8pt}{4pt}

% Custom list formatting
\setlist[itemize]{leftmargin=15pt, itemsep=2pt, parsep=0pt, topsep=4pt}
\setlist[itemize,2]{leftmargin=20pt, itemsep=1pt, parsep=0pt, topsep=2pt}

% Hyperlink setup
\hypersetup{
	colorlinks=true,
	linkcolor=blue,
	urlcolor=blue,
	citecolor=blue
}

\begin{document}
\begin{center}
  {\Huge\bfseries Dr. Fernando Guti\'errez Canales}\\[8pt]
  \faMapMarker*~Göttingen, Germany \quad | \quad
  \faPhone~+33 6 51 70 80 80 \quad | \quad
  \faEnvelope~\href{mailto:carl.cfgc@gmail.com}{carl.cfgc@gmail.com} \quad | \quad
  \faGitlab~\href{https://gitlab.obspm.fr/fgutierrez}{gitlab.obspm.fr/fgutierrez} \quad | \quad
  \faGithub~\href{https://github.com/Fernando-Canales}{github.com/Fernando-Canales}
\end{center}
	
	\vspace{10pt}
	
	% Two-column layout
	\begin{multicols}{2}
		
		% LEFT COLUMN
		\section{Skills}
		
		\subsection{Programming:}
		\begin{itemize}
			\item \textbf{Advanced:} \textbf{Python} (NumPy, SciPy, Matplotlib, Conda, Venv, Scikit-Learn), C, Fortran,
			\item \textbf{Intermediate:} R, Version Control (Git and Subversion), bash, ssh
			\item \textbf{Beginner:} Parallel computing: OpenMP (used with Fortran and C)
			\item \textbf{Computational modeling:} PDEs, Large Data-bases, Time-series, Data analysis, Data visualization
			\item \textbf{Astrophysics:} Hydrodynamics, kinetic simulations, Bayesian statistics
			\item \textbf{Office applications:} Microsoft Office, Libre Office, \LaTeX , Vim, nano
			\item \textbf{Languages:} Spanish (native), English (C1), French (good) and German (basic)
		\end{itemize}
		
		\subsection{Interpersonal/Soft:}
		
		\begin{itemize}
			\item Effective Team Collaboration
			\item Strategic Planning and Problem Solving
			\item Leading and Delegation (including mentoring junior colleagues)
			\item Clear Communication (both technical and non-technical)
			\item Adaptability and Time Management
		\end{itemize}
		
		% RIGHT COLUMN
		\section{Summary}
		PhD in Astrophysics with 5+ years of experience in data modeling and visualization, statistical analysis, and software development. Built scalable Python-based data pipelines for space mission analysis (PLATO, ESA). Skilled in working with large datasets, Bayesian modeling, and machine learning. Effective communicator with a strong record of collaboration in international research environments. Now seeking to apply these skills to data-driven roles in industry, including analytics, finance, or tech.
		
		\section{Research Experience \& Education}
		
		
		\subsection{March 2022 – March 2025}
		\textbf{PhD | Paris Observatory and Max Planck Institute for Solar System Research}
		\begin{itemize}
			\item \textbf{Thesis:} The \mbox{PLATO space mission:} Double-aperture photometry and Centroid Shifts to detect False Positives
			\begin{itemize}
				\item Grade: \textbf{With honors}
			\end{itemize}
			\item \textbf{Key result:} Developed python-based simulations to estimate the detection efficiency of exoplanet vetting techniques (the established centroid shift and the novel double-aperture photometry) for the PLATO space mission. Integrated C and bash libraries; published code on Gitlab. The produced results constitute the first estimation of the overall efficiency of both techniques.
			\item Collaborated with an international consortium ($<$20 institutions and countries) on instrument development and pipeline design for ESA's PLATO space mission.
		\end{itemize}
		
		\subsection{2019 - 2021}
		\textbf{Master's degree in Sciences: Astrophysics | University of Guanajuato, Mexico}
		\begin{itemize}
		\item \textbf{GPA: 9.5/10}
		\item \textbf{Thesis:} Homogeneous Analysis of K2 exoplanet systems hosting USP planets
		\item \textbf{Key result:} Implemented and mastered the scientific software pyaneti to improve exoplanet parameter estimation for K2 systems. 
		\end{itemize}
		
		\subsection{2014 – 2019}
		\textbf{Bachelor's Degree in Physics | University of Guanajuato, Mexico}
		\begin{itemize}
			\item \textbf{GPA: 9.0/10}
			\item \textbf{Undergraduate Research Assistant} in the group of Non-linear Optics of the University of Guanajuato | 01/2016 – 10/2019
			\item \textbf{Thesis:} Atomic theory and scientific realism
			\item \textbf{Key result:} Developed a scientific publication about the most important philosophical and physical ways to show that atoms exist. 
		\end{itemize}
		
		\subsection{Research intern | ESTEC (European Space Agency), The Netherlands | 2023}
		\begin{itemize}
			\item Estimating Charge Transfer Inefficiency, CTI, parameters for PLATO detectors using Python and DS9.
			\item Conducted in-situ measurements with a real PLATO CCD detector
		\end{itemize}
	\end{multicols}
	
	\section{Conferences}

		\begin{itemize}
			\item EAS (European Astronomical Union) | 2024
			\item PLATO Week \# 15 Meeting | 2024
			\item PLATO Week \# 14 Meeting | 2023
			\item Workshop \textit{Journé des thèses} | 2022
		\end{itemize}

	
	\section{Scholarships \& Awards}
	\begin{itemize}
		\item PhD obtained with Honors | 2025
		\item Erasmus$+$ scholarship for an internship at ESTEC, the largest European Space Agency (ESA) center in Europe | 2023
		\item Master's degree obtained with Honors | 2021
		\item Scholarship for studying a master's degree in Mexico with international competence | 2018
		\item Bachelor's degree obtained with Honors | 2019
	\end{itemize}
	
	\section{Publications}
	\begin{itemize}
		\item Interpretation of Optical and IR Light Curves for Transitional Disk Candidates in NGC 2264 Using the Extincted Stellar Radiation and the Emission of Optically Thin Dust Inside the Hole, 2021, E. Nagel, F. Gutiérrez-Canales, S. Morales-Gutiérrez and A. P. Sousa.
		\item The young HD 73583 (TOI-560) planetary system: Two 10-M$_{\oplus}$ mini-Neptunes transiting a 500-Myr-old, bright, and active K dwarf, 2023, O. Barrag\'an,.., F. Gutiérrez-Canales, ..., E. Nagel
		\item Detecting False Positives with PLATO using Double-Aperture Photometry and Centroid Shifts , F. Gutiérrez-Canales, R. Samadi A. Birch, submitted in December 2024.
	\end{itemize}
	
\end{document}