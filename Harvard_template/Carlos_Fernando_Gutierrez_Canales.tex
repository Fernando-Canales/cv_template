\documentclass[10pt,a4paper]{article}
\usepackage[margin=1in]{geometry}
\usepackage{hyperref}
\usepackage{enumitem}
\usepackage{titlesec}
\usepackage{parskip}
\usepackage{fontawesome5}  % for icons (optional)

\titleformat{\section}{\large\bfseries}{}{0em}{}[\titlerule]

\begin{document}

\begin{center}
    {\Huge \textbf{Dr. Fernando Gutiérrez Canales}}\\
    Göttingen, Germany \quad | \quad +33 6 51 70 80 80 \quad | \quad \href{mailto:carl.cfgc@gmail.com}{carl.cfgc@gmail.com} \\
    \href{https://gitlab.obspm.fr/fgutierrez}{GitLab: gitlab.obspm.fr/fgutierrez} \quad | \quad \href{https://github.com/Fernando-Canales}{GitHub: github.com/Fernando-Canales}
\end{center}

\vspace{1em}

\section*{Professional Summary}
PhD-trained Data Scientist and Python Developer with 5+ years of experience building robust, scalable data pipelines and analytical tools in scientific research environments. Skilled in automating data workflows, designing metrics and monitoring systems, and collaborating across technical and customer-facing teams. Proven ability to turn complex datasets into reliable, maintainable solutions. Eager to contribute to mission-driven organizations like Cozero, where data innovation drives sustainability impact.

\section*{Technical Skills}
\textbf{Programming:} Python (NumPy, SciPy, Pandas, Matplotlib, Scikit-learn, Astropy), C, Fortran, Bash, R and Shell scripting \\
\textbf{Tools \& Workflow:} Git, Subversion, Docker, Conda, Jupyter, DS9 \\
\textbf{Data Pipelines:} Modular ETL workflows, Data cleaning/monitoring, Custom simulation pipelines \\
\textbf{Data Science \& Analytics:} A/B Testing, Machine Learning, Time-Series Analysis, Bayesian Inference, Monte Carlo Methods and Large Datasets \\
\textbf{AI \& Automation:} LLM-based tools (in progress), AI-assisted analysis, Python automation \\
\textbf{Software \& Productivity:} LaTeX, Microsoft Office, LibreOffice, Vim and nano \\
\textbf{Languages:} Spanish (native), English (C1), French (good) and German (basic) 


\section*{Soft Skills}
\begin{itemize}[leftmargin=1.5em]
    \item Cross-functional Collaboration in International Teams
    \item Technical Project Management \& Agile Practices
    \item Clear Communication of Analytical Results
    \item Mentorship and Leadership in Scientific Programming
    \item Creative Problem Solving with Noisy or Incomplete Data
\end{itemize}

\section*{Experience}

\textbf{PhD Researcher} \\
\textit{Paris Observatory \& Max Planck Institute for Solar System Research} \\
Mar 2022 -- Mar 2025
\begin{itemize}[leftmargin=1.5em]
    \item Developed Python-based simulation pipeline to estimate detection efficiency for planetary transit vetting methods (centroid shifts and double-aperture photometry) in the PLATO space mission.
    \item Integrated C and Bash modules; published results as part of the PLATO international science consortium.
    \item Collaborated with over 15 institutions in mission-critical software development and instrument calibration.
\end{itemize}

\textbf{Research Intern} \\
\textit{ESTEC (European Space Agency), The Netherlands} \\
Summer 2023
\begin{itemize}[leftmargin=1.5em]
    \item Conducted in-situ CCD measurements for PLATO detector calibration.
    \item Estimated Charge Transfer Inefficiency (CTI) parameters using DS9 and Python analysis scripts.
\end{itemize}

\section*{Education}

\textbf{PhD in Astrophysics} \\
\textit{Paris Observatory \& MPS Göttingen} \\
With Honors (Expected 2025)
\begin{itemize}
	\item Thesis: The PLATO mission: Detecting False Positives using Double-aperture photometry and Centroid Shifts
\end{itemize}

\vspace{0.5em}

\textbf{MSc in Astrophysics} \\
\textit{University of Guanajuato, Mexico} \\
GPA: 9.5/10, With Honors (2021)

\begin{itemize}[leftmargin=1.5em]
    \item Thesis: Homogeneous Analysis of K2 planetary systems hosting USP planets
\end{itemize}

\textbf{BSc in Physics} \\
\textit{University of Guanajuato, Mexico} \\
GPA: 9.0/10, With Honors (2019)

\section*{Conferences \& Publications}
\begin{itemize}[leftmargin=1.5em]
    \item Presented at EAS (2024), PLATO Weeks \#14 \& \#15, Journée des Thèses (2022)
    \item Co-author on peer-reviewed publications on planetary systems and photometric signal analysis
    \item Submitted: Gutierrez-Canales et al., \textit{Detecting False Positives with PLATO using Double-Aperture photometry and Centroid Shifts}, Dec 2024
\end{itemize}

\section*{Awards}
\begin{itemize}[leftmargin=1.5em]
    \item Erasmus+ Fellowship -- ESTEC Internship (2023)
    \item Full Scholarships \& Honors Degrees (2019, 2021, 2025)
\end{itemize}

\end{document}
