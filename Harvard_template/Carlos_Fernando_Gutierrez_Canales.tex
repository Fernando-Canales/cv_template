\documentclass[10pt,a4paper]{article}
\usepackage[margin=1in]{geometry}
\usepackage{hyperref}
\hypersetup{
    colorlinks=true,
    linkcolor=blue,
    urlcolor=blue,
    citecolor=blue
}
\usepackage{enumitem}
\usepackage{titlesec}
\usepackage{parskip}
\usepackage{fontawesome5}  % for icons (optional)

\titleformat{\section}{\large\bfseries}{}{0em}{}[\titlerule]

\begin{document}

\begin{center}
    {\Huge \textbf{Fernando Gutiérrez Canales}}\\
    {\Large Dr. (Astrophysics) | Data Scientist}\\    
    Göttingen, Germany \quad | \quad \href{https://www.linkedin.com/in/fernando-guti%C3%A9rrez-canales-804876343/}{Linkedin} \quad | \quad +33 6 51 70 80 80 \quad | \quad \href{mailto:carl.cfgc@gmail.com}{carl.cfgc@gmail.com} \\
    \href{https://gitlab.obspm.fr/fgutierrez}{GitLab: gitlab.obspm.fr/fgutierrez} \quad | \quad \href{https://github.com/Fernando-Canales}{GitHub: github.com/Fernando-Canales}
\end{center}

\vspace{1em}

\section*{Professional Summary}
Data Scientist and Scientific Developer with a PhD in Astrophysics and 5+ years of experience building scalable data pipelines for frontier scientific research. My work focuses on creating robust software solutions and analytical tools to manage, process, and extract meaning from complex, multi-source datasets. I have a proven track record of successful collaboration within large international consortia, turning research needs into practical, maintainable software. I am eager to contribute my skills in data management and analysis to support the impactful, mission-driven research at the German Primate Center. 

\section*{Work Experience}
\textbf{Paris Observatory \& Max Planck Institute for Solar System Research} \, \textbf{Meudon and Göttingen}
\textit{PhD researcher} \hfill \textit{Mar 2022 -- Jul 2025}
\begin{itemize}[leftmargin=1.5em]
    \item Developed robust Python-based simulation pipeline to estimate cutting-edge detection efficiency for planetary transit methods (centroid shifts and double-aperture photometry) in the PLATO space mission.
    \item Integrated C and Bash modules; published results as part of the PLATO international science consortium.
    \item Collaborated with over 15 institutions around the world in mission-critical software development and instrument calibration.
\end{itemize}

\textbf{ESTEC (European Space Agency)} \hfill \textbf{Noordwijk, The Netherlands}\\
\textit{Research Intern}  \hfill \textit{June 2023 -- August 2023}
\begin{itemize}[leftmargin=1.5em]
    \item Conducted in-situ CCD measurements for PLATO detector calibration.
    \item Automated CTI parameter estimation using Python, improving calibration efficiency for ESA detectors
\end{itemize}

\section*{Personal Projects}
\textbf{Scientific \& Technology Development Projects} \hfill \textbf{León, Mexico} \\
\textit{Math \& Science Teacher for at-risk high school students} \hfill \textit{2021-2022}
\begin{itemize}
\item Developed engaging data-driven lessons that increased student performance by 60\,\%
\item Included pedagogical materials in my math and physics lessons, increasing the engagement and participation of the students
\end{itemize}
\textit{Digitalization \& Data Analysis} \hfill \textit{2021-2022}
\begin{itemize}
\item Digitalized and structured patient records into a searchable database, improving management and accessibility for clinic staff.
\item Engineered a Python tool utilizing Spotify’s API to analyze listening habits, build a recommendation model, and generate personalized, data-driven playlists.
\end{itemize}

\section*{Skills \& Interests}
\textbf{Hard Skills:} Python (NumPy, SciPy, Pandas, Matplotlib, Scikit-learn, Astropy), C, Fortran, Bash, R, Shell scripting, Git, Subversion, Docker, Unix, Linux, Conda, Jupyter, DS9, SQL, Custom simulation pipelines, Machine Learning, Time-Series Analysis, Bayesian Inference, Monte Carlo Methods, Large Datasets, LLM-based tools (in progress), AI-assisted analysis, Tableau, LaTeX, Microsoft Office, LibreOffice, Vim, nano, Web development: HTML, CSS \\

\textbf{Soft Skills:} Problem-Solving, Teamwork, Time Management, Adaptability, Curiosity, Self-Learning, Resilience, Analytical Thinking, Technical Communication, Project Management, Creativity and Innovation, Attention to Detail

\textbf{Languages:} 
\begin{itemize}
 \item Spanish (native) 
 \item English (C1)
 \item French (B2)
 \item German (basic)
\end{itemize} 

\textbf{Interests:} Astronomy, Arts, Poetry, Sports, Wellness.

\section*{Education}

\textbf{PhD in Astrophysics} \\
\textit{Paris Observatory \& MPS Göttingen} \\
With Honors
\begin{itemize}
	\item Thesis: The PLATO mission: Detecting False Positives using Double-aperture photometry and Centroid Shifts
\end{itemize}

\vspace{0.5em}

\textbf{MSc in Astrophysics} \\
\textit{University of Guanajuato, Mexico} \\
GPA: 9.5/10, With Honors (2021)

\begin{itemize}[leftmargin=1.5em]
    \item Thesis: Homogeneous Analysis of K2 planetary systems hosting USP planets
\end{itemize}

\textbf{BSc in Physics} \\
\textit{University of Guanajuato, Mexico} \\
GPA: 9.0/10, With Honors (2019)

\section*{Scientific Outreach Activities}
\textbf{Astronomía Divertida Group} \hfill \textbf{Guanajuato, Mexico}\\
\textit{Astronomy Department University of Guanajuato} \hfill \textit{2021-2022}
\begin{itemize}
\item Participation and organization of astronomy talks in public spaces, including observations of the night sky at archeological sites
\item Participation in astronomy outreach talks in the local TV channel of Guanjuato 
\end{itemize}

\textbf{IX REA (Mexican Astronomy Students Meeting)} \hfill \textbf{Cancún, Mexico (Hybrid)}\\
\textit{Quintana Roo Planetarium} \hfill \textit{Summer 2021}

\begin{itemize}
\item Participated as a SOC member and a moderator for scientific talks at the IX REA

\end{itemize}

\section*{Research activities and awards}
\begin{itemize}[leftmargin=1.5em]
    \item Presented an electronic poster about my PhD work at the European Astronomical Society, EAS (2024), Padova.
    \item Presented a scientific talk about my PhD work at the PLATO Week \#14, (2023), Prague.
    \item Presented a poster about my PhD work at Journée des Thèses (2022), Meudon.
    \item Co-author on peer-reviewed publications on planetary systems and photometric signal analysis
    \item Full Scholarships \& Degrees with honors (2019, 2021, 2023, 2025)
    \end{itemize}

\end{document}
